\chapter{Implementation}
\label{ch:implementation}

% Describe the architecture of your implementation and explain the choices you made.
% Why did you use library X instead of Y?

In this chapter, we present our implementation of the different components that we used to conduct our experiments.
We will first describe the setup and configuration of the Kali machines, as well as the use of the open-source SCION implementation.
Subsequently, in \cref{sec:impl:SCIONpackets}, we will explain the processes for generating and modifying SCION packets.
\cref{sec:impl:attackerModel} details the implementation of the different attack models.
Finally, \cref{sec:impl:VolumetricDoS} explains how we conducted the volumetric denial-of-service attacks.

\section{Kali Machines}

The Kali virtual machines mentioned in \cref{sec:setup} contain numerous tools and software for performing penetration tests.
We upgraded these machines to include all the additionally needed programs that were listed in \cref{sec:analysis-methods-tools}.

To interact with SCION, we installed the SCION end host stack on the Kali machines.
This stack includes a general \texttt{scion} helper binary, the SCION dispatcher, and the SCION daemon, all of which we built from the open-source code.
The \texttt{scion} binary is a tool that facilitates interaction with the SCION network.
It can be used to send SCMP echo requests (i.e, SCION pings) or perform traceroutes to a specific destination.
The binaries require configuration files to connect to the SCION network.
For example, the addresses and interfaces of the AS-local border routers must be known, or the TRC of the whole ISD is required.
We extracted the necessary configuration files from the Anapaya SCION devices, copied them to the Kali machines, and attempted to run the SCION end host stack.
This attempt was unsuccessful because Anapaya uses a different format for the configuration files than the open-source SCION implementation.
After some manual adjustments, we managed to get the dispatcher and daemon running and connected to the SCION production network.

At this point, we were capable of sending and receiving SCION packets to and from any other SCION device in the network.
Additionally, this setup allowed us to fetch all possible path information to a specific destination using the \texttt{scion} binary with the \texttt{scion showpaths} command.


\section{Open-Source SCION}
We utilize the open-source code not only to build the end host stack, but also to verify the behavior of the Anapaya's proprietary SCION implementation.
By leveraging the open-source code, we can validate certain hypotheses and determine if specific features are implemented identically or modified by Anapaya.


\section{SCION Packets}
\label{sec:impl:SCIONpackets}
Typically, SCION packets are not directly accessible to the user as they are generated and sent by a SCION application, such as the \texttt{scion} binary.
However, for our experiments, we needed to generate and modify custom SCION packets.
The Python library \texttt{scapy}, with its SCION extension, facilitates this process.
It allows stacking different headers on top of each other and defining every field of them.
This capability enabled us to generate SCION packets with custom payloads and headers.
To send and receive SCION packets, \texttt{scapy} does not require a running SCION dispatcher or daemon.
Nonetheless, in our implementation, we use the SCION end host stack to fetch the path information, which we then use to generate SCION packets with valid path headers.




\section{Attacker Models}
\label{sec:impl:attackerModel}

This section outlines the implementation of the different attacker models.
Their individual capabilities are described in \cref{sec:attacker-models}.

\subsection{Malicious SCION End Host}
We used the Kali machines to simulate a malicious SCION end host.
These machines are integrated into the SCION network and are capable of sending and receiving SCION packets.
This enables us to generate and dispatch custom, potential maliciously crafted, SCION packets and to other SCION devices, using the approach mentioned previously in \cref{sec:impl:SCIONpackets}.
Since the Kali machines can also receive and analyze SCION packets, we can verify whether the packets are correctly received and processed by the target devices.
This process allows us to assess whether the target devices are vulnerable to the specific attack.
Additionally, we also have access to our target devices, including all three operational Anapaya SCION devices, to evaluate the impact of potential attacks.

Moreover, the Kali machines are used to fetch and store path information for future use.
This is necessary because path information is frequently updated and removed, despite the paths remaining valid and not yet expired.
By preserving these path information, we can later send SCION packets with valid path headers (i.e., with correct MAC values) but with potentially expired hop fields.


\subsection{On-path Attacker}
A logical approach to implementing an on-path attacker would involve utilizing the three operational Anapaya SCION devices at the CYD Campus locations.
These devices are present on each path between our SCION end hosts (i.e., Kali machines) and any other reachable SCION device within the network.
This would enable us to intercept, monitor, block, replay, and modify packets in transit.
However, employing these devices would affect other SCION users and their connections.
Due to ethical and legal concerns, we opted against using the Anapaya devices for our experiments.
Moreover, even in the absence of these concerns, implementing an on-path attacker on these devices would be challenging, particularly due to the need for custom software installation.

Instead, again the Kali machines are used to simulate an on-path attacker.
We configure a socket through which an end host can send and receive SCION packets.
On the same machine, a custom Python script operates on the other side of the socket to simulate the on-path attacker.
In the absence of an attack, this script simply forwards received SCION packets to the border router and sends the corresponding response back to the end host via the socket.
During an attack, the script has the capability to modify, save, or drop both incoming and outgoing SCION packets.


\subsection{Off-path Attacker}

The CYD Campus SCION setup, comprising three distinct locations and three separate ASes, facilitates the simulation of an off-path attacker.
In this setup, one location is designated as the attacker, while the other two serve as potential victims.
The attacker is capable of sending SCION packets to the victims but lacks direct access to the communication path between the two.
This way, we can realistically implement an off-path attacker within the SCION production network.
Typically, a Kali machine is employed to simulate an end-host off-path attacker.
In certain scenarios, we also consider the Anapaya device within the attacker's AS as malicious, thereby simulating a fully malicious AS that operates off-path.

\subsection{Non-SCION Adversary}
To simulate an attacker without SCION access, we could utilize any device that is not connected to the SCION network.
For this purpose, we select a device connected to the traditional Internet and located outside the CYD Campus.
ETH Zürich provided access to their network as well as a dedicated server from which we can launch attacks.
An advantage of this setup is that both ETH Zürich and the CYD Campus in Zürich share the same Internet service provider, SWITCH.
This allows for simpler coordination of specific attacks, such as volumetric denial-of-service attacks.


\section{Volumetric Attacks}
\label{sec:impl:VolumetricDoS}
This section details the implementation of volumetric denial-of-service attacks to evaluate the resilience of SCION under high traffic loads.
We employ TRex, a state-of-the-art traffic generator, to produce the high volume of traffic required for these attacks.
The setup, located at ETH Zürich, consists of a server equipped with an MT27800 Mellanox 100 Gbit/s network interface card, running TRex version 2.87.
This allowed us to conduct the following two types of volumetric attacks:
We conducted two types of volumetric attacks:
One from a traditional (non-SCION) Internet adversary and another from a malicious SCION end host.

\begin{itemize}
    \item \textbf{Traditional Internet:}
    To simulate a volumetric attack originating from the non-SCION adversary, we configured TRex to continuously send large UDP packets towards the public IP of the CYD Campus location in Zürich.
    \item \textbf{SCION:} Since TRex does not natively support generating SCION packets, we developed a custom script that creates a valid SCION packet, including the required path information to the border router at CYD Campus Zürich.
    This script stores the packet in a PCAP file, which can then be replayed by TRex.
    Similar to the first attack, the SCION packet contains a large payload to maximize network resource consumption.
\end{itemize}

We were permitted to generate up to 20 Gbit/s of traffic from the ETH network, which exceeds the 10 Gbit/s capacity of the CYD Campus link in Zürich, ensuring that the attack traffic fully utilizes the available bandwidth.

To evaluate the impact of the attack on SCION's performance, we employed two SCION-specific tools.
First, we utilize \texttt{scion ping} to assess the reachability of the targeted SCION AS during the attack.
Second, to measure the remaining available bandwidth, we use the \texttt{scion-bwtester} tool.
This provides insights into the degree of resource exhaustion caused by the attack.

We chose Thun as the measurement point due to its superior network infrastructure compared to Lausanne, offering a significantly higher bandwidth capacity (10 Gbit/s vs. 1 Gbit/s).
This selection ensures that any observed performance degradation could be more directly attributed to the DoS attack rather than limitations in the network infrastructure.