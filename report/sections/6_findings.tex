\chapter{Findings}
\label{ch:findings}

% * Repeatable, specific description
% * Provide intuition, key idea
% * State what the problem is that you want to address in that section
% * Describe all parts of life cycle: setup, run, maintenance, etc.
% * Explain with an example (together with the intuition and description, the scheme should have been explained 3 times from different perspectives)
% * Did we really fully explore the design space?
%     * Why are simpler solutions inappropriate?
%     * Explore design alternatives!
%     * Provide intuition on why certain design choices were taken, why alternatives were rejected
%     * Describe a strawman approach, show why it's inadequate


\section{SCION Protocol}






\section{Anapaya Router}



\subsection{Vulnerabilities}

\subsubsection{Docker}
% libc6 libssl1.1 openssl 1.1.1k

The individual services of the Anapaya router are containerized using Docker, and we analyzed these containers to identify potential vulnerabilities.
The Docker images are based on Google's distroless Debian 11.1 base image \cite{githubGitHubGoogleContainerToolsdistroless}.
Such distroless images only contain the necessary dependencies and does not include a full operating system.
This reduces the attack surface and makes the images more secure.
However, it was found that this base image still contains unused binaries, such as \texttt{openssl}, its corresponding library \texttt{libssl}, and \texttt{c\_rehash}.
Furthermore, these binaries are outdated and vulnerable to various security threats, such as command injection, use-after-free, double free, and denial of service attacks.
To improve the security of these Docker containers, it is recommended to remove these unused binaries by changing the base image to the no-ssl variant of the distroless image (see \cite{githubGitHubGoogleContainerToolsdistroless}).

We also analyzed the libraries that are actually used to run the container and found that \texttt{libc} is outdated and has some vulnerabilities.
We were not able to directly exploit these vulnerabilities, but it is recommended to update it to the latest version to prevent potential security threats.





\subsubsection{Host System}



\subsection{Ubuntu Compliance}
We found that the Anapaya router is not fully compliant with the Ubuntu security guidelines.
Numbers in parentheses in this section refer to the corresponding section in the CIS Ubuntu Linux 22.04 LTS audit.
Major issues include the following:
\begin{itemize}
    \item On two of the three examined routers, there is no password or other re-authentication needed to escalate privileges to root (5.2.4).
    \item While the systems are configured to only accept at least 6 characters long passwords, there is no password policy in place. For example, password complexity is not enforced (5.3.3.2.3) and also password reuse is not restricted (5.3.3.3.1).
    \item There is no shell timeout configured, which would log out an inactive user after a certain time (5.4.3.2). By setting a timeout, the risk of unauthorized access to the system can be reduced and also frees up resources associated with the idle session.
\end{itemize}


\subsection{Appliance}
No authentication for CYD SCION user to access the Appliance.
Only requires knowledge of IP of the border router.
Enables change of Border router configuration, such as:
\begin{itemize}
    \item change firewall rules
    \item read and modify SCION forwarding key (used in the MAC calculation for example)
    \item change DNS and NTP servers
    \item add new TRC
    \item Disable router completely
    \item Add own authentication and lock out other users
\end{itemize}


\subsection{Web Server}
The appliance is also being served by a web server, and runs on the router.
It is accessible by any AS-local SCION user or any remote user residing in an AS where a SIG connection is configured.
The web server used is Caddy, version 2.6.4, dating back to February 2023.
This version is vulnerable to a denial of service attack, that consumes the routers resources and eventually crashes it, taking down all its services, including SCION!
This attack is called Rapid Reset and works as follows:
It uses the HTTP/2 protocol and its feature to open multiple streams within a single TCP connection.
The attacker misuses this feature by opening a large number of streams at the same time and instantly closing them again.
By immediately closing the streams after opening them, the attacker ensures that they never exceed the maximum number of streams allowed by the server.
However, the server still needs to perform substantial work for canceled requests, including allocating memory for a new stream, parsing the query, performing header decompression, and more \cite{googleWorksNovel}.

After modifying the code in this proof of concept attack code \cite{githubGitHubMicrictorhttp2rststream}, we were able to attack our Anapaya router and finally taking it down withing a few seconds.
The attack even works when the Appliance is configured with authentication, as the web server is not protected by it.

The solution to this problem is to update the Caddy web server to version 2.7.5 or later, which prevents this Rapid Reset attack \cite{githubReleasesCaddyservercaddy}.




\subsection{MAC}
Check if MAC algorithm is the same as the one that can be found in the open source version.
For this, we first extracted the master key from the file system and derived a MAC key according to definitions of the open source code.
With this derived MAC key we calculated the MAC value for a given hop field.
Our calculation result matched the original MAC value, which proves the usage of the open source MAC algorithm.
This algorithm is a CMAC and can be considered secure in the sense of unforgeability.
During a discussion with professor Perrig we found out that the master secret should be refreshed every day to ensure good cryptographic hygiene.
This, however, is not the case on the Anapaya routers, and it always uses the same value.


