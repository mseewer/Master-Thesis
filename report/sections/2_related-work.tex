\chapter{Related Work}
\label{ch:related-work}

% This is different from the background:
% While the background contains systems that you build on, the related work contains systems that try to solve the same or a similar problem as the thesis.
% * Analyze research landscape, how does the work fit in?
% * Present work that others may think to be related and explain why it isn't related

In this chapter we review literature and previous work related to our security analysis.
We will also justify the need for our work by showing how it fits into the research landscape and how it differs from existing work.

In 2016, one of the first looking at the security of SCION was Rothenberger \cite{Rothenberger2016}.
At that time, SCION was still in its early stages of development and had not yet reached the level of maturity it has today.
His analysis compared SCION to other competing protocol offering some kind of security improvements over the traditional Internet.
Furthermore, the thesis also identified various vulnerabilities such as memory leaks, resource exhaustion, misconfigurations, and cryptographic weaknesses.
The findings were fixed, and also his cryptographic suggestions were implemented in the protocol making SCION more secure and robust.
Given SCION's significant evolution since then, a new security analysis is now essential to assess its current resilience.


The Bachelor thesis of Lehmann \cite{Lehmann2021} in 2021 focused on the data plane security aspects of SCION.
His research mainly focused on the temporal lensing and AS spoofing attacks in SCION.
Lehmann also implemented these attacks in SCIONLab, a testbed for SCION, to demonstrate their feasibility.
The thesis also proposed countermeasures (e.g. source authentication) to mitigate these attacks.


So far, the security analyzes of SCION have mainly focused on the data plane or the protocol itself.
This change with recent works that also looked at the secure deployment of SCION.

Hager's work \cite{Hager2024} took a general approach and showed the security guarantees and limits of SCION in three different deployment scenarios.
It identified various attack vectors, such as the hardware, the operating system, the SCION network management, or the different network connection in the scenarios.
For the latter Hager evaluated the probability of various network attacks (e.g. packet sniffing, DDoS, Man-in-the-Middle, route hijacking, AS spoofing) and their extent of damage in the different scenarios.
Furthermore, the information security was being evaluated for the different deployments, where the focus was on the confidentiality, integrity, availability, as well as anonymity of the SCION network traffic.
As this work was in cooperation with the federal department of defense, the work also included suggestions for the government on how to deploy SCION securely.


A few years back Maurer \cite{Maurer2021} from the Swiss Stock exchange (SIX) looked at the security of the operational devices.
Additionally, a forensic framework was developed which acts as a knowledge base for future forensic investigations on the SCION systems.
This work is closely related to our analysis, as we also focus on the security and configurations of the operational devices here at CYD Campus.
Nevertheless, a lot changed in the last few years in the SCION ecosystem, and thus a new security analysis is needed to assess the current security guarantees.



Last but not least I want to mention the practical work of Niederer \cite{Niederer2022}, which among other things analyzed and accompanied the setup of the commercial SCION devices at the CYD Campus.
Thanks to his work, we were able to now focus on the security of SCION in the setting of the CYD Campus.

