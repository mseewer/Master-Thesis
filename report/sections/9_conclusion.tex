\chapter{Conclusion}
\label{ch:conclusion}

% * What have you learnt from the research?
% * What are surprising findings from the work?
% * What future work is possible?
% * Some additional insights.
% * *The conclusion should be very different from the abstract!*

This thesis set out to critically assess the security of the SCION Internet architecture, with a particular focus on Anapaya's proprietary implementation.
The investigation was driven by the central question:
Can SCION, in its current form as deployed by Anapaya, truly fulfill its promise of a more secure, resilient, and reliable Internet architecture?

To address this question, the thesis conducted a comprehensive analysis across three key areas:

First, an in-depth examination of the deployed Anapaya EDGE devices was performed, revealing several weaknesses.
The study uncovered insecure default configurations, such as inadequate SSH settings and the presence of unused and vulnerable software packages.
More alarmingly, the SCION management system lacked authentication, posing a significant risk of unauthorized access and control over the network infrastructure.
Even when authentication was enabled, the management system remained susceptible to attacks capable of crashing the device within seconds and thereby also taking down all SCION services.


The second key area focused on examining the SCION protocol implementation to assess the security of the deployed version and also to identify potential derivations from the open-source version.
The analysis revealed significant flaws, with Anapaya's implementation, including the absence of source authentication within the SCION protocol and the use of persistent master secrets.
These vulnerabilities enable attackers to forge packets, leading to various attack vectors that undermine the integrity and trustworthiness of the network.


The third and final area of investigation evaluated the impact of the traditional Internet on SCION by conducing a real-world volumetric denial-of-service attack.
\TODO{Mention vol dos test}

\\
It became evident that while SCION provides a robust framework, the security of its implementation heavily depends on the rigor of operational practices and the constant evolution of security measures.
The vulnerabilities identified in this thesis highlight the need for continuous security audits, particularly as SCION and its implementations are increasingly deployed in critical infrastructure.

In answering the thesis's key question, this research concludes that while SCION has the potential to fulfill its promise as a next-generation Internet architecture, the current implementation by Anapaya falls short in several critical areas.
The identified weaknesses, if left unaddressed, could undermine the security and reliability that SCION aims to provide.
However, these challenges are not insurmountable.
The findings offer a clear roadmap for improving the security of SCION deployments, emphasizing the need for stronger source authentication mechanisms, more secure device configurations, and \TODO{vol dos}.

Ultimately, this research contributes to the ongoing development and maturation of SCION, providing valuable insights that can help shape its future evolution.
By addressing the vulnerabilities identified in this thesis, SCION can move closer to realizing its full potential as a secure and resilient Internet architecture for the modern digital world.


\section{Future Work}

While our analysis focused on Anapaya EDGE devices, future research could extend to analyzing Anapaya CORE and GATE devices.
This expansion would help identify differences among the products and uncover potential security issues within them.
Given the rapid development of SCION and Anapaya's products, it would also be beneficial to monitor how the security posture of these devices evolves over time.

Further examination of Anapaya's proprietary SCION components could reveal additional security vulnerabilities.
This could be achieved through reverse engineering of these proprietary elements or by collaborating with Anapaya to gain deeper insights into their implementation.

An in-depth investigation of the SCION control plane presents another promising area for future research.
Control plane issues can significant impact SCION, as it underpins many of its features.
This thesis did not fully cover the control plane in the operational CYD Campus environment due to the associated risks and potential unintended consequences.
Moreover, the control plane is not directly exposed to end users, making it harder for a single end host attacker to exploit.
However, key areas for future work include CP-PKI, certificate management, or path discovery mechanism.

Future research could also explore the development and security of new SCION applications.
This includes how these applications should handle critical scenarios, such as receiving unauthenticated SCMP error messages or managing requests with expired hop fields.

Our work involved a volumetric denial-of-service attack from the traditional Internet and evaluating its impact on SCION.
It would be interesting to assess the impact of other attacks originating from the traditional Internet.
Additionally, future studies could evaluate the feasibility and the potential effects of these attack when happening directly within the SCION network.

Finally, future studies could investigate SCION traffic sniffing and traffic analysis.
This would help determine what information a passive eavesdropper could collect by intercepting SCION traffic.
Such analysis would be particularly relevant for privacy-sensitive applications that depend on SCION for secure communication.

